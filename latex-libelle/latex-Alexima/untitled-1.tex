\documentclass[a4paper,12pt]{article}
\usepackage[utf8]{inputenc}
\usepackage{geometry}
\geometry{top=2cm, bottom=2cm, left=2.5cm, right=2.5cm}
\usepackage{graphicx}
\usepackage{hyperref}
\usepackage{amsmath}
\usepackage{enumitem}
\usepackage{listings}

	\title{Projet de réalisation d'une Application Météorologique Web/Mobile }
\author{}
\date{}

\begin{document}

\maketitle

\vfill % Crée de l'espace flexible avant le texte
\begin{center}
    \Huge \textbf{Libellé} % Augmenter la taille du texte
\end{center}
\vfill % Crée de l'espace flexible après le texte, pour centrer verticalement

\begin{center}
    \textbf{Remis à} \\[1em] % Texte avant les noms
    \textbf{Sanlove Eden ALEXIMA} \\
    \textbf{Jackyno JEAN-BAPTISTE} \\
    \textbf{Phillipe JOSE}
\end{center}

	
\begin{center}
    \today
\end{center}


\clearpage % Saut de page après le titre

\tableofcontents % Table des matières sur une nouvelle page
\newpage % Saut de page après la table des matières

\section{Contexte et objectif du projet}
\subsection{Présentation du projet}
Cette application a pour but de fournir des données météorologiques en temps réel ainsi que des prévisions à des utilisateurs pour des usages variés . 
Elle permettra également de consulter des données historiques et d'accéder à des alertes météorologiques importantes. Les utilisateurs doivent pouvoir accéder à ces informations de manière simple et intuitive. Une application hybride accessible via un navigateur web et une application mobile (Android/iOS).

\subsection{Objectifs principaux}
Les principaux objectifs de l'application sont :
\begin{itemize}
    \item Afficher les prévisions météo à court et long terme.
    \item Offrir un accès aux données météorologiques historiques.
    \item Proposer des cartes interactives pour visualiser les prévisions.
    \item Fournir des notifications en temps réel pour les alertes météo.
\end{itemize}

\section{Public cible}
\subsection{Définir les utilisateurs}
Les utilisateurs ciblés par cette application sont :
\begin{itemize}
    \item \textbf{Utilisateurs professionnels} : météorologues, organisations spécialisées dans le climat, agriculteurs, pilotes, capitaines de navires, etc.
    \item \textbf{Utilisateurs grand public} : voyageurs, passionnés de météo, etc.
    \item \textbf{Administrateurs} : gestionnaires des données et des utilisateurs(UrGEO)
\end{itemize}

\section{Fonctionnalités attendues}
\subsection{Fonctionnalités principales pour l'utilisateur}
\begin{itemize}
    \item Accès aux prévisions météo par localisation géographique (ville, région, pays).
    \item Consultation des prévisions horaires et journalières.
    \item Utiliser des API de cartes telles que "Leaflet", "Google Maps" ou "Mapbox" pour afficher des cartes interactives montrant les stations météorologiques et leurs relevés.
    \item Affichage des données météo en temps réel (température, précipitations, humidité, vent, etc.).
    \item Consultation des données historiques pour une période donnée.
    \item Notifications push pour les alertes météo (tempêtes, orages, etc.).
     \item Intégrer des annotations ou des événements clés (ex. une tempête ou un pic de température) sur les graphiques pour un meilleur suivi.
\item Ajouter des couches de cartes spécifiques (par ex. affichage de la vitesse du vent, des zones de précipitations, ou des températures par zones colorées).
\item Permettre un filtrage basé sur des critères spécifiques (afficher uniquement les zones avec des températures supérieures à 30°C, par exemple).

\end{itemize}

\subsection{Fonctionnalités avancées}
\begin{itemize}
    \item Personnalisation des alertes (types d’alertes, régions spécifiques).
    \item Personnalisation des données météorologique
    \item Utilisation de graphiques pour visualiser les tendances météo.
    \item Partage des prévisions sur les réseaux sociaux.
 

\end{itemize}

\subsection{Fonctionnalités administratives}
\begin{itemize}
    \item Gestion des utilisateurs et des permissions.
    
    \item Interface de gestion des capteurs
    \begin{itemize}
        \item Ajout de nouveaux capteurs
        \item Surveillance des capteurs
        \item Configuration des capteurs
        \item Type de données collectées sur le capteur
	\item Configuration des seuils d'alerte
        \item Indicateurs de statut pour chaque capteur
     \item Historique des données / Variations
    \item Alerte en temps réel(email, notifications, SMS)
	\item Jourrnal des erreurs pour surveiller les disfonctionnements ou les pannes
\item Tableau de bord de tous les capteurs 
    
\end{itemize}
\end{itemize}

\section{Fonctionnalités envisagées}
\subsection{ Fonctionnalité optionnelle : Contrôle à distance d'un climatiseur}

\textbf{Description :} \\
L'application pourrait intégrer une fonctionnalité de contrôle à distance d'un climatiseur, déclenchée automatiquement lorsque la température enregistrée par les capteurs dépasse un certain seuil (défini par l'utilisateur ou configuré par défaut). Cette fonctionnalité viserait à maintenir une température ambiante agréable sans intervention manuelle.

\textbf{Objectif :} 
\begin{itemize}
    \item Garantir le confort de l'utilisateur en ajustant automatiquement la température de l'environnement via un dispositif de climatisation.
    \item Contribuer à la gestion énergétique en fonction des conditions climatiques mesurées.
\end{itemize}

\textbf{Fonctionnement prévu :}
\begin{enumerate}
    \item \textbf{Détection de la température :} Les capteurs mesurent en temps réel la température ambiante et envoient les données à l'application.
    \item \textbf{Analyse des données :} Lorsque la température dépasse un certain seuil (à définir par l'utilisateur), une alerte est générée.
    \item \textbf{Action de contrôle :} L'application enverrait un signal de contrôle au climatiseur (via un protocole approprié comme IR, Wi-Fi, ou API spécifique), demandant une mise en marche ou une modification des paramètres (comme baisser la température).
    \item \textbf{Retour d'état :} Le climatiseur renverrait un retour d'état (si compatible) pour indiquer que l'action a été exécutée.
\end{enumerate}

\textbf{Pré-requis techniques :}
\begin{itemize}
    \item Un climatiseur compatible avec une interface de commande à distance (ex : via infrarouge, Wi-Fi, ou autre protocole supporté).
    \item Une API ou un protocole de communication permettant l'intégration avec l'application.
\end{itemize}

\textbf{Étude de faisabilité :} \\
Cette fonctionnalité nécessite une \textit{étude de faisabilité} qui sera menée pendant le projet. Les aspects suivants devront être étudiés :
\begin{itemize}
    \item \textbf{Compatibilité matérielle :} Identification des climatiseurs compatibles (par exemple, modèles équipés de Wi-Fi, d'IR, ou d'une API).
    \item \textbf{Protocole de communication :} Vérification de l'existence de solutions logicielles et matérielles pour permettre le contrôle à distance (via infrarouge, Wi-Fi, etc.).
    \item \textbf{Limites techniques :} Analyse des éventuelles limitations liées aux modèles de climatiseurs disponibles et à leur capacité à être commandés à distance.
\end{itemize}

\textbf{Évaluation des risques :}
\begin{itemize}
    \item \textbf{Incompatibilité des climatiseurs :} Certains modèles de climatiseurs ne peuvent pas être contrôlés à distance via une API standard.
    \item \textbf{Complexité de l'intégration :} L'intégration pourrait nécessiter des protocoles spécifiques ou des équipements supplémentaires (comme des récepteurs infrarouges).
    \item \textbf{Coût supplémentaire :} L'achat de matériels spécifiques pour la communication ou la commande des climatiseurs pourrait entraîner des coûts supplémentaires.
\end{itemize}

\textbf{Décision de mise en œuvre :} \\
En fonction des résultats de l'étude de faisabilité, cette fonctionnalité sera soit :
\begin{itemize}
    \item \textbf{Implémentée} si elle est jugée techniquement et économiquement viable,
    \item \textbf{Abandonnée ou reportée} dans le cas contraire.
\end{itemize}

\section{Technologies utilisées}
\subsection{Frontend (côté utilisateur)}
\begin{itemize}
    \item Web : Ionic \&  Angular.
    \item Mobile : Ionic et Angulars.
  \item Langage : JavaScript.

\end{itemize}

\subsection{Backend (côté serveur)}
\begin{itemize}
    \item Langage : C\#.
    \item Base de données : MongoDB.
    \item API : ASP .NET Core.
\end{itemize}

\subsection{Infrastructure}
\begin{itemize}
    \item Hébergement sur le cloud (AWS).
    \item Notifications push via Firebase Cloud Messaging.
\end{itemize}
\subsection{Autres outils}
\begin{itemize}
  \item Versionnage(Github).
  \item Suivi et evaluation(JIRA)
\item Test endpoint(Postman)
\end{itemize}

\section{Interface utilisateur (UI/UX)}
\begin{itemize}
    \item Design simple et intuitif.
    \item Accès facile aux informations essentielles (température, précipitations, etc.).
    \item Couleurs adaptées à la météo (ex. : bleu pour le ciel, gris pour les nuages, etc.).
    \item Responsive design pour s'adapter aux écrans mobiles, tablettes, et ordinateurs.
\end{itemize}

\section{Contraintes techniques et fonctionnelles}
\begin{itemize}
    \item Compatibilité avec iOS, Android, et les principaux navigateurs web.
    \item Sécurité des données utilisateurs (ex. : localisation) et confidentialité des informations personnelles.
    \item Performance : l'application doit être rapide et fluide, même avec une connexion faible.
 \item Connectivité : Etude pour une meilleure reseau de connectivité des capteurs avec l'API (WIFI/Bluetooth/Zigbee/Z-wave/Cellulaire).
 \item Protocole: Etude du protocole de communication adequat (HTTP/HTTPS/MQTT/Websocket) .
 \item Configuration des capteurs
\end{itemize}

\section{Planification du projet}
\subsection{Phases du projet}
\begin{enumerate}
    \item Analyse des besoins et spécifications détaillées.
    \item Développement du prototype (wireframes et maquettes).
   \item Etude des capteurs
    \item Développement frontend et backend.
    \item Intégration des API météorologiques.
    \item Tests (unitaires, fonctionnels, utilisateurs).
    \item Mise en production et maintenance.
\end{enumerate}

\subsection{Délais et Budget}
\begin{itemize}
    \item Estimation délai: 8 mois 
    \item Budget nécessaire selon les ressources (développeurs, serveurs, outils).
\end{itemize}

\section{Critères de validation}
\begin{itemize}
    \item Validation des fonctionnalités par des tests utilisateurs.
    \item Conformité aux exigences et fonctionnalités définies.
    \item Performance et fiabilité (temps de réponse, stabilité, etc.).
\end{itemize}

\section{Maintenance et évolutions futures}
\subsection{Suivi}
La maintenance de l'application comprendra des mises à jour régulières pour corriger les bugs et assurer sa pérennité.

\subsection{Évolutions}
De nouvelles fonctionnalités pourront être ajoutées, comme l'intégration de nouvelles données météo ou des prévisions saisonnières.

\end{document}
