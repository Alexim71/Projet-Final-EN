\documentclass[a4paper,12pt]{article}
\usepackage[utf8]{inputenc}
\usepackage{geometry}
\geometry{top=2cm, bottom=2cm, left=2.5cm, right=2.5cm}
\usepackage{graphicx}
\usepackage{hyperref}
\usepackage{amsmath}
\usepackage{enumitem}
\usepackage{listings}

\title{Cahier des Charges\\Application Météorologique Web/Mobile}
\author{Sanlove Eden ALEXIMA}
\date{\today}

\begin{document}

\maketitle

\tableofcontents
\newpage

\section{Contexte et objectif du projet}
\subsection{Présentation du projet}
Cette application a pour but de fournir des données météorologiques en temps réel ainsi que des prévisions à des utilisateurs pour des usages variés (Urgeo, agriculture, tourisme, navigation, etc.). 
Elle permettra également de consulter des données historiques et d'accéder à des alertes météorologiques importantes. Les utilisateurs doivent pouvoir accéder à ces informations de manière simple et intuitive. Une application hybride accessible via un navigateur web et une application mobile (Android/iOS).

\subsection{Objectifs principaux}
Les principaux objectifs de l'application sont :
\begin{itemize}
    \item Afficher les prévisions météo à court et long terme.
    \item Offrir un accès aux données météorologiques historiques.
    \item Proposer des cartes interactives pour visualiser les prévisions.
    \item Fournir des notifications en temps réel pour les alertes météo.
\end{itemize}

\section{Public cible}
\subsection{Définir les utilisateurs}
Les utilisateurs ciblés par cette application sont :
\begin{itemize}
    \item \textbf{Utilisateurs professionnels} :météorologues,organisations spécialisées dans le climat, agriculteurs, pilotes, capitaines de navires, etc.
    \item \textbf{Utilisateurs grand public} : voyageurs, passionnés de météo, etc.
    \item \textbf{Administrateurs} :Urgeo, gestionnaires des données et des utilisateurs.
\end{itemize}

\section{Fonctionnalités attendues}
\subsection{Fonctionnalités principales pour l'utilisateur}
\begin{itemize}
    \item Accès aux prévisions météo par localisation géographique (ville, région, pays).
    \item Consultation des prévisions horaires et journalières.
    \item Visualisation sur une carte (intégration avec Google Maps ou autre).
    \item Affichage des données météo en temps réel (température, précipitations, humidité, vent, etc.).
    \item Consultation des données historiques pour une période donnée.
    \item Notifications push pour les alertes météo (tempêtes, orages, etc.).
\end{itemize}

\subsection{Fonctionnalités avancées}
\begin{itemize}
    \item Personnalisation des alertes (types d’alertes, régions spécifiques).
    \item Utilisation de graphiques pour visualiser les tendances météo.
    \item Partage des prévisions sur les réseaux sociaux.
  \item Partage des prévisions sur les réseaux sociaux.

\end{itemize}

\subsection{Fonctionnalités administratives}
\begin{itemize}
    \item Gestion des utilisateurs et des permissions.
    \begin{itemize}
    \item Interface de gestion des Capteurs
    \begin{itemize}
        \item Ajout de nouveaux capteurs
        \item Surveillance des capteurs
        \item Configuration des capteurs
        \item Type de donnes collectees sur le capteur
	\item Configuration des seuils d'alerte
        \item Indicateurs de statut pour chaque capteur
     \item Historique des donnees / Variations
    \item Alerte en temps reel(email, notifications, SMS)
	\item Jourrnal des erreurs pour surveiller les disfonctionnements ou les pannes
\item Tableau de bord de tous les capteurs 
    \end{itemize}
\end{itemize}
\end{itemize}

\section{Technologies utilisées}
\subsection{Frontend (côté utilisateur)}
\begin{itemize}
    \item Web : Ionic \&  Angular.
    \item Mobile : Ionic et Angulars.
  \item Langage : JavaScript.

\end{itemize}

\subsection{Backend (côté serveur)}
\begin{itemize}
    \item Langage : C\#.
    \item Base de données : MongoDB.
    \item API : ASP .NET Core.
\end{itemize}

\subsection{Infrastructure}
\begin{itemize}
    \item Hébergement sur le cloud (AWS).
    \item Notifications push via Firebase Cloud Messaging.
  \item Versionnage(Github).
\end{itemize}

\section{Interface utilisateur (UI/UX)}
\begin{itemize}
    \item Design simple et intuitif.
    \item Accès facile aux informations essentielles (température, précipitations, etc.).
    \item Couleurs adaptées à la météo (ex. : bleu pour le ciel, gris pour les nuages, etc.).
    \item Responsive design pour s'adapter aux écrans mobiles, tablettes, et ordinateurs.
\end{itemize}

\section{Contraintes techniques et fonctionnelles}
\begin{itemize}
    \item Compatibilité avec iOS, Android, et les principaux navigateurs web.
    \item Sécurité des données utilisateurs (ex. : localisation) et confidentialité des informations personnelles.
    \item Performance : l'application doit être rapide et fluide, même avec une connexion faible.
 \item Connectivite: Etude pour une meilleure reseau de connectivite des capteurs avec l'API(WIFI/Bluetooth/Zigbee/Z-wave/Cellulaire).
 \item Protocole: Etude du protocole de communication(HTTP/HTTPS/MQTT/Websocket).
 \item Configuration des capteurs
\end{itemize}

\section{Planification du projet}
\subsection{Phases du projet}
\begin{enumerate}
    \item Analyse des besoins et spécifications détaillées.
    \item Développement du prototype (wireframes et maquettes).
    \item Développement frontend et backend.
    \item Intégration des API météorologiques.
    \item Tests (unitaires, fonctionnels, utilisateurs).
    \item Mise en production et maintenance.
\end{enumerate}

\subsection{Délais et Budget}
\begin{itemize}
    \item Estimation des délais pour chaque phase du projet.
    \item Budget nécessaire selon les ressources (développeurs, serveurs, outils).
\end{itemize}

\section{Critères de validation}
\begin{itemize}
    \item Validation des fonctionnalités par des tests utilisateurs.
    \item Conformité aux exigences et fonctionnalités définies.
    \item Performance et fiabilité (temps de réponse, stabilité, etc.).
\end{itemize}

\section{Maintenance et évolutions futures}
\subsection{Suivi}
La maintenance de l'application comprendra des mises à jour régulières pour corriger les bugs et assurer sa pérennité.

\subsection{Évolutions}
De nouvelles fonctionnalités pourront être ajoutées, comme l'intégration de nouvelles données météo ou des prévisions saisonnières.

\end{document}
